%%%%%%%%%%%%%%%%%%%%%%%%%%%%%%%%%%%%%%%%%%%%%%%%%%%%%%%%%%%%%%%%%%%%%%%%%%%%%%%%
\section{Digital Signatures}
\label{sec:pke}

A digital signature scheme $\DS = (\kg,\sign,\ver)$ is a triple of
algorithms. Key generation is randomized and outputs a key pair $(\pk,\sk)$,
where $\pk$ is called the public or verification key and $\sk$ is called the secret or
signing key.
Signing takes as input a secret key $\sk$ and a message $M$ and outputs a
signature, usually denoted $\sigma$.
Verification takes as input a public key $\pk$, message $M$, and signature
$\sigma$ and outputs a
bit.



\fpage{.20}{
		\underline{$\UFCMA_\DS^\advA$}\\
    $(\pk,\sk) \getsr \kg$\\
    $(M^*,\sigma^*) \getsr \advA^\SignOracle(\pk)$\\
    If $M^* \in \calM$ then \\
    \myInd Ret $\false$\\
		Ret $\ver(\pk,M^*,\sigma^*)$\medskip

    \underline{$\SignOracle(M)$}\\
    $\calM \gets \calM \cup \{M\}$\\
    $\sigma \getsr \sign(\sk,M)$\\
    Ret $\sigma$
	}

\bnm
  \AdvUFCMA{\DS}{\advA} = \Prob{\UFCMA_{\DS}^\advA\Rightarrow\true}
\enm



Let $\DS$ be the full domain hash (FDH) digital signature scheme. 


\begin{theorem*}
Let $\DS$ be the FDH scheme using $\RSAk$ and $\Horacle\Colon\msgspace\rightarrow\Z_N^*$ modeled as
a RO. Let $\advA$ be any $\UFCMA_\DS$-adversary making at most $q_h$ queries to
$\Horacle$ and $q_s$ queries to its signing oracle. 
Then we give an $\RSAk$-adversary $\advB$ such that
\bnm
    \AdvUFCMA{\DS}{\advA} \le (q_h + q_s + 1) \cdotsm\AdvOWF{\RSAk}{\advB}
\enm
Adversary~$\advB$ runs in time that of $\advA$ plus $\bigO(q_s+q_h)$. 
\end{theorem*}


\begin{proof}
To start we make some simplifying assumptions about $\advA$. First, whenever it
makes a query to $\SignOracle$ on a message $M$ it has previously made a query
$\Horacle(M)$. Second, when it outputs a forgery attmept $(M^*,\sigma^*)$ it has
previously queried $M^*$. And, as we normally assume, it does not repeat a query
to $\Horacle$.  Should $\advA$ not abide by these restrictions, we can easily
build an adversary $\advA'$ from $\advA$ that does so, at cost at most an extra
$q_s + 1$ queries to $\Horacle$.

Now, to build some intuition, let's first consider the case that $q_s = 0$, that is, 
we are arguing about unforgeability in a no-message attack. 
%Assume that $\advA$
%queries $\Horacle(M^*)$ where $M^*$ is the output message for its forgery. This
%is without loss. 
Then in this case we can 
guess which random oracle query corresponds to the solution, and program its
output to be equal be the OWF challenge $Y$.  In more detail, we set our forgery
adversary $\advB$ to be:
%Intuitively, to forge against FDH one needs to invert RSA on $\Horacle(M)$ for
%some $M$. Then we can build an inverter $\advB$ that works as follows. 

\fpage{.25}{
\underline{$\UFCMA_{\DS}$}\\
$((N,e),(N,d)) \getsr \kg$\\
$(M^*,\sigma^*) \getsr \advA^{\Horacle,\SignOracle}((N,e))$\\
If $M^* \in \calM$ then Ret $\false$\\
Ret $\left(\TabH[M^*] = (\sigma^*)^e \bmod N\right)$\medskip

\underline{$\Horacle(M)$}\\
If $\TabH[M] = \bot$ then\\
\myInd $\TabH[M] \getsr \Z_N^*$\\
Ret $\TabH[M]$\medskip

\underline{$\SignOracle(M)$}\\
$\calM \gets \calM \cup \{M\}$\\
$X \getsr \Horacle(M)$\\
Ret $X^d \bmod N$
}

\fpage{.25}{
\underline{$\G_0$ \;\;\; \fbox{$\G_1$}}\\
$((N,e),(N,d)) \getsr \kg$\\
$i^* \getsr [1,q]$\\
$i \gets 0$\\
$(M^*,\sigma^*) \getsr \advA^{\HashSim}((N,e))$\\
If $(M^* \ne M_{i^*})$ then \\
\myInd $\badtrue$\\
\myInd \fbox{Ret $\false$}\\
Ret $\left(\TabH[M^*] = (\sigma^*)^e \bmod N\right)$\medskip

\underline{$\HashSim(M)$}\\
$i \gets i+1$\\
$M_i \gets M$\\
If $i = i^*$ then\\
\myInd Ret $\TabH[M_i] \getsr \Z_N^*$\\
$\TabH[M_i] \getsr \Z_N^*$\\
Ret $\TabH[M_i]$
}
\fpage{.25}{
\underline{$\G_2$}\\
$((N,e),(N,d)) \getsr \kg$\\
$i^* \getsr [1,q]$\\
$i \gets 0$\\
$(M^*,\sigma^*) \getsr \advA^{\HashSim}((N,e))$\\
If $(M^* \ne M_{i^*})$ then \\
\myInd $\badtrue$\\
\myInd \fbox{Ret $\false$}\\
Ret $\left(\TabH[M^*] = (\sigma^*)^e \bmod N\right)$\medskip

\underline{$\HashSim(M)$}\\
$i \gets i+1$\\
$M_i \gets M$\\
If $i = i^*$ then\\
\myInd Ret $\TabH[M_i] \gets Y$\\
$\TabH[M_i] \getsr \Z_N^*$\\
Ret $\TabH[M_i]$
}

\fpage{.25}{
\underline{$\advB_{toy}((N,e),Y)$}\\
$(M^*,\sigma^*) \getsr \advA^{\HashSim}((N,e))$\\
Ret $\sigma^*$\medskip

\underline{$\HashSim(M)$}\\
If $i = i^*$ then\\
\myInd Ret $Y$\\
}


\fpage{.30}{
\underline{$\advB((N,e),Y)$}\\
$i^* \getsr [1,q]$\\
$i \gets 0$\\
$(M^*,\sigma^*) \getsr \advA^{\HashSim,\SignSim}((N,e))$\\
If $(M^* \ne M_{i^*})$ then $X' \getsr \Z_N^*$\\
Else $X' \gets \sigma^*$\\
Ret $X'$\medskip

\underline{$\HashSim(M)$}\\
$i \gets i+1$\\
$M_i \gets M$\\
If $i = i^*$ then\\
\myInd Ret $Y$\\
$\sigma_i \getsr \Z_N^*$\\
$\TabH[M_i] \gets (\sigma_i)^e \bmod N$\\
Ret $\TabH[M_i]$\medskip

\underline{$\SignSim(M)$}\\
Let $i$ be s.t.~$M_i = M$\\
If $i = i^*$ then\\
\myInd Ret $\sigma \getsr \Z_N^*$\\
Ret $\sigma_i$
}
\fpage{.30}{
\underline{$\G_0$}\\
$((N,e),(N,d) \getsr \kg$\\
$X \getsr \Z_N^*$\\
$Y \gets X^e \bmod N$\\
$i^* \getsr [1,q]$\\
$i \gets 0$\\
$(M^*,\sigma^*) \getsr \advA^{\HashSim,\SignSim}((N,e))$\\
If $(M^* \ne M_{i^*})$ then \\
\myInd $\badtrue$\\
\myInd $X' \getsr \Z_N^*$\\
$X' \gets \sigma^*$\\
Ret $(X = X')$\medskip

\underline{$\HashSim(M)$}\\
$i \gets i+1$\\
$M_i \gets M$\\
If $i = i^*$ then\\
\myInd Ret $Y$\\
$\sigma_i \getsr \Z_N^*$\\
$\TabH[M_i] \gets (\sigma_i)^e \bmod N$\\
Ret $\TabH[M_i]$\medskip

\underline{$\SignSim(M)$}\\
Let $i$ be s.t.~$M_i = M$\\
If $i = i^*$ then\\
\myInd $\badtrue$\\
\myInd Ret $\sigma \getsr \Z_N^*$\\
Ret $\sigma_i$
}
\fpage{.30}{
\underline{$\G_1$}\\
$((N,e),(N,d) \getsr \kg$\\
$X \getsr \Z_N^*$\\
$Y \gets X^e \bmod N$\\
$i^* \getsr [1,q]$\\
$i \gets 0$\\
$(M^*,\sigma^*) \getsr \advA^{\HashSim,\SignSim}((N,e))$\\
If $(M^* \ne M_{i^*})$ then \\
\myInd $\badtrue$\\
\myInd $X' \gets \sigma^*$\\
$X' \gets \sigma^*$\\
Ret $(X = X')$\medskip

\underline{$\HashSim(M)$}\\
$i \gets i+1$\\
$M_i \gets M$\\
If $i = i^*$ then\\
\myInd Ret $Y$\\
$\sigma_i \getsr \Z_N^*$\\
$\TabH[M_i] \gets (\sigma_i)^e \bmod N$\\
Ret $\TabH[M_i]$\medskip

\underline{$\SignSim(M)$}\\
Let $i$ be s.t.~$M_i = M$\\
If $i = i^*$ then\\
\myInd $\badtrue$\\
\myInd Ret $\sigma \gets X$\\
Ret $\sigma_i$
}


\begin{align*}
  \AdvUFCMA{\DS}{\advA} &= \Prob{\G_1\Rightarrow\true}\\
      &\le \Prob{\G_0\Rightarrow\true} + \Prob{\bad_0}\\
      &= \AdvOWF{\RSAk}{\advB} + \Prob{\bad_0}
\end{align*}

\begin{align*}
\AdvOWF{\RSAk}{\advB_{toy}} = \AdvUFCMA{\DS}{\advA}
\end{align*}

Intuitively $\advB$ wins as long as $\advA$ wins and $i^*$ is the correct guess
of which hash query by $\advA$ corresponds to the winning $M^*$. (Recall that we
are assuming that $\advA$ always queries $\Horacle$ for the forgery message
$M^*$.) To analyze this, let game $\G_0$ be equal to $\UFCMA_{\DS}^\advA$ except
that it additionally includes a random choice of $i^* \getsr [1,q]$ and sets a
flag $\bad$ to true should $M^* \ne M_{i^*}$. Let
game $\G_1$ be the same as $\G_0$ except that  it chooses a random index $i^*
\getsr [1,q]$ and checks if $M^*$ is equal to the $i\thh$ random oracle
query. If not, it sets a flag $\bad$ to true and outputs $\false$, 
Otherwise it checks if $\advA$'s output is a winning forgery, outputing
$\true$ if so and $\false$ otherwise. Let $\good$ be the event that $\bad$ is
not set in game $\G_0$, and we let $\good$ be the same event for $\G_1$ (a
slight abuse of notation). Then we have that both games are
identical-until-$\bad$ and a variant of the fundamental lemma of game playing
says that:
\bnm
  \Prob{\G_0 \Rightarrow\true \land\good} = \Prob{\G_1\Rightarrow\true \land
  \good} \;.
\enm
Then using this, we have that:
\begin{align*}
\AdvOWF{\RSAk}{\advB} 
  &= \Prob{\G_0\Rightarrow\true}
  &\ge \Prob{\G_0\Rightarrow\true\land\good}\\
  &= \Prob{\G_1\Rightarrow\true\land\good}\\
  &= \Prob{\G_1\Rightarrow\true}\cdot\Prob{\good}\\
  &= \AdvUFCMA{\DS}{\advA}\cdotsm\frac{1}{q}
\end{align*}
%Given that $\Horacle$ is modeled as a random oracle, this would seem
%to correspond to having to 
%We must create an adversary $\advB$ that inverts RSA .

\end{proof}


