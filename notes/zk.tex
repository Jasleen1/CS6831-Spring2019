%%%%%%%%%%%%%%%%%%%%%%%%%%%%%%%%%%%%%%%%%%%%%%%%%%%%%%%%%%%%%%%%%%%%%%%%%%%%%%%%
\section{Zero-Knowledge and Proofs of Knowledge}
\label{sec:idprots}


Following Boneh-Shoup's treatment, 
an effective relation is a binary relation $\calR \subseteq \calX \times \calY$
where $\calR, \calX,\calY$ are sets. (We assume the are efficiently
recognizable, meaning one can determine if a value is a member of the set
efficiently.) Elements of $\calY$ are called statements and elements of $\calX$
are called witnesses.


\fpage{.25}{
\underline{$\NIZK_{\Phi,\Simu}^\advA$}\\[1pt]
$b \getsr \bits$\\
$b' \getsr \advA^{\ProofOracle}$\\
Return $(b = b')$\medskip

\underline{$\ProofOracle(x,Y)$}\\[1pt]
If $b = 1$ then\\
\myInd $\pi \getsr \Gen(x,Y)$\\
If $b = 0$ then \\
\myInd $\pi \getsr \Simu(Y)$\\
Return $\pi$
}

\fpage{.15}{
\underline{$\NIZK1_{\Phi,\Simu}^\advA$}\\[1pt]
$b \getsr \bits$\\
$b' \getsr \advA^{\ProofOracle,\Horacle}$
Return $(b = b')$\medskip

\underline{$\ProofOracle(x,Y)$}\\[1pt]
$\pi \getsr \Gen^\Horacle(x,Y)$\\
Return $\pi$\medskip

\underline{$\Horacle(M)$}\\[1pt]
If $\TabH[M] = \bot$ then\\
\myInd $\TabH[M] \getsr \bits^n$\\
Return $\TabH[M]$
}
\fpage{.15}{
\underline{$\NIZK1_{\Phi,\Simu}^\advA$}\\[1pt]
$b \getsr \bits$\\
$b' \getsr \advA^{\ProofOracle,\HashSim}$\\
Return $(b = b')$\medskip

\underline{$\ProofOracle(x,Y)$}\\[1pt]
$\pi \getsr \Simu_P(Y)$\\
Return $\pi$\medskip

\underline{$\HashSim(M)$}\\[1pt]
Ret $\Simu_H(M)$
}

A zero-knowledge (ZK) proof is 

For our purposes, an identification protocol is a tuple of algorithms
$(\IDkeygen,\IDcom,\IDresp,\IDver)$ and a set $\IDchalset$ called the challenge space. 
The key generation $\IDkeygen$ outputs a public key and secret key pair
$(\pk,\sk)$. The commitment generator $\IDcom$ is randomized and takes the
public key $\pk$ as input. It outputs a commitment. The response generator $\IDresp$ is
deterministic, takes as input the secret key $\sk$, the commitment $R$, and the
challenge $c \in \IDchalset$, and outputs a response~$z$. The verification
algorithm $\IDver$ is deterministic, takes as input a public key $\pk$, 
commitment $R$, challenge $c$, and response $z$, and outputs a bit.  

\bnm
\AdvIDPASS{\IDscheme,q_{id}}{\advA} = \Prob{\IDPASS_{\IDscheme,q_{id}}^\advA\Rightarrow\true}
\enm

\fpage{.40}{
\underline{$\IDPASS_{\IDscheme,q_{id}}^\advA$}\\[1pt]
$(\pk,\sk)\getsr \IDkg$\\
$\queried \gets \false$\\
For $i = 1$ to $q_{id}$ do\\
\myInd $R_i \getsr \IDcom(\pk)$\\
\myInd $c_i \getsr \IDchalset$\\
\myInd $z_i \getsr \IDresp(\sk,R_i,c_i)$\\
$z^* \getsr \advA^{\VerOracle}(\pk,(R_1,c_1,z_1),\ldots,(R_{q_{id}},c_{q_{id}},z_{q_{id}}))$\\
Ret $\IDver(\pk,R^*,c^*,z^*)$\medskip

\underline{$\VerOracle(R^*)$}\\[1pt]
If $\queried = \true$ then Ret $\bot$\\
$\queried \gets \true$\\
$c^*\getsr \IDchalset$\\
Ret $c^*$
}


\fpage{.43}{
\underline{$\advB(X)$}\\[1pt]
$\queried \gets \false$\\
For $i = 1$ to $q_{id}$ do\\
\myInd $z_i \getsr \Z_q$\\
\myInd $c_i \getsr \Z_q$\\
\myInd $R_i \getsr g^{z_i}X^{-c_i}$\\
$\coins \getsr \CoinSpace_\advA$\\
$z^* \gets \advA^{\VerOracle}(\pk,(R_1,c_1,z_1),\ldots,(R_{q_{id}},c_{q_{id}},z_{q_{id}}) \semi \coins)$\\
$z^{**} \gets \advA^{\VerOracle}(\pk,(R_1,c_1,z_1),\ldots,(R_{q_{id}},c_{q_{id}},z_{q_{id}}) \semi \coins)$\\
If $c^* = c^{**}$ then Ret $x \getsr \Z_q$\\
$x \gets (z^* - z^{**}) / (c^* - c^{**})$\\
Ret $x$\medskip

\underline{$\VerOracle(R^*)$}\\[1pt]
If $\queried = \false$ then \\
\myInd $\queried \gets \true$\\
\myInd Ret $c^{*} \getsr \Z_q$\\
Ret $c^{**} \getsr \Z_q$
}


\begin{lemma*}
Let $S$ and $T$ be finite, non-empty sets and let $f\Colon S\times T\rightarrow
\bits$ be a function. Let $\calX,\calY,\calY'$ be independent random variables,
with $\calX$ taking values in $T$ and $\calY,\calY'$ being uniformly chosen from
$T$. Let $\Pr[f(\calX,\calY) = 1] = \epsilon$. Then 
\bnm
  \Prob{f(\calX,\calY) = 1 \land f(\calX,\calY') = 1 \land \calY \ne \calY'} 
      \ge \epsilon^2 - \frac{\epsilon}{|T|}
\enm
\end{lemma*}

\begin{proof}[Sketch]
Let $\epsilon(s) = \Pr[f(s,\calY) = 1]$. 
Then $\Ex[\epsilon(\calX)] = \epsilon$. Let $N_s$ be the number of values
$Y \in T$ such that $f(s,Y) = 1$. Let $\calU_s$ be the event that $f(s,\calY) =
1 \land f(s,\calY') = 1 \land \calY \ne \calY'$. Then
\bnm
  \Prob{\calU_s} = \frac{N_s(N_s - 1)}{|T|^2} = \epsilon(s)^2 - \frac{\epsilon(s)}{|T|} 
\enm
Let $\calU$ be event that $f(\calX,\calY) = 1 \land f(\calX,\calY') = 1 \land
\calY \ne \calY'$. Then 
\begin{align*}
  \Prob{\calU} 
  %& =\sum_{s \in S} \Prob{f(\calX,\calY) = 1 \land f(\calX,\calY') = 1 \land \calY \ne \calY' \land \calX = s}\\
  & = \sum_{s \in S} \Prob{f(s,\calY) = 1 \land f(s,\calY') = 1 \land \calY \ne \calY' \land \calX = s}\\
  & = \sum_{s \in S} \Prob{f(s,\calY) = 1 \land f(s,\calY') = 1 \land \calY \ne \calY'}\Prob{\calX = s}\\
  & = \sum_{s \in S} \Prob{\calU_s} \Prob{\calX = s}\\
  & = \sum_{s \in S} \left(\epsilon(s)^2 - \frac{\epsilon(s)}{|T|}\right) \Prob{\calX = s}\\
  & = \Ex[\epsilon(\calX)^2] - \frac{\Ex[\epsilon(\calX)]}{|T|}\\
  & \ge \Ex[\epsilon(\calX)]^2 - \frac{\Ex[\epsilon(\calX)]}{|T|}\\
  & = \epsilon^2 - \frac{\epsilon}{|T|}
\end{align*}
\end{proof}

\begin{proof}[Sketch]
Let $\calU$ be event that $f(\calX,\calY) = 1 \land f(\calX,\calY') = 1 \land
\calY \ne \calY'$. Then 
\begin{align*}
  \Prob{\calU} 
  %& =\sum_{s \in S} \Prob{f(\calX,\calY) = 1 \land f(\calX,\calY') = 1 \land \calY \ne \calY' \land \calX = s}\\
  & = \sum_{s \in S} \Prob{f(s,\calY) = 1 \land f(s,\calY') = 1 \land \calY \ne \calY' \land \calX = s}\\
  & = \sum_{s \in S} \Prob{f(s,\calY) = 1 \land f(s,\calY') = 1 \land \calY \ne \calY'}\Prob{\calX = s}\\
  & = \sum_{s \in S} \Prob{\calU_s} \Prob{\calX = s}\\
  & = \sum_{s \in S} \left(\epsilon(s)^2 - \frac{\epsilon(s)}{|T|}\right) \Prob{\calX = s}\\
  & = \Ex[\epsilon(\calX)^2] - \frac{\Ex[\epsilon(\calX)]}{|T|}\\
  & \ge \Ex[\epsilon(\calX)]^2 - \frac{\Ex[\epsilon(\calX)]}{|T|}\\
  & = \epsilon^2 - \frac{\epsilon}{|T|}
\end{align*}

Let $\epsilon(s) = \Pr[f(s,\calY) = 1]$. 
Then $\Ex[\epsilon(\calX)] = \epsilon$. Let $N_s$ be the number of values
$Y \in T$ such that $f(s,Y) = 1$. Let $\calU_s$ be the event that $f(s,\calY) =
1 \land f(s,\calY') = 1 \land \calY \ne \calY'$. Then
\bnm
  \Prob{\calU_s} = \frac{N_s(N_s - 1)}{|T|^2} = \epsilon(s)^2 - \frac{\epsilon(s)}{|T|} 
\enm
\end{proof}




\begin{theorem*}
Let $G$ be a cyclic group, $q_{id} \ge 0$, 
and $\advA$ be an $\IDPASS_{\IDscheme,q_{id}}$-adversary for the Schnorr
scheme~$\IDscheme$
built using $G$. Then we give an $\DL_G$-adversary $\advB$ such that
\bnm
  \AdvIDPASS{\IDscheme,q_{id}}{\advA} \le \sqrt{\AdvDL{G}{\advB}} + \frac{1}{|G|} \;.
\enm
Adversary~$\advB$ runs in time at most twice that of $\advA$ plus
$\bigO(q_{id})$.
\end{theorem*}


\begin{theorem*}
Let $G$ be a cyclic group and $\DS$ be the Schnorr digital signature scheme
using random oracle $\Horacle \Colon\msgspace\rightarrow\Z_{|G|}$.  
Let $\advA$ be an $\UFCMA_{\DS}$-adversary making at most $q_h$ RO queries and
$q_s$ signing queries.
Then we give an $\DL_G$-adversary $\advB$ such that
\bnm
  \AdvUFCMA{\DS}{\advA} \le \frac{q_s(q_s+q_h+1)}{|G|} + (q_h+1)\left(\sqrt{\AdvDL{G}{\advB}} + \frac{1}{|G|}\right) \;.
\enm
Adversary~$\advB$ runs in time at most twice that of the sum of the running time
of~$\advA$ and $\bigO(q_h+q_s)$.
\end{theorem*}
